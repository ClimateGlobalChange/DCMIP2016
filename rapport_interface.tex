% !TEX encoding = UTF-8 Unicode

\documentclass{article}   
\usepackage[american]{babel} 
\usepackage[utf8]{inputenc}  
\usepackage[T1]{fontenc}
\usepackage{amssymb,amsmath}
\usepackage{graphicx}
\usepackage{siunitx}
                    
\title{Dynamical Core Model Intercomparaison Project (DCMIP) Test Case Interface Documment}                  % Titre du document
\author{AntoninVerlet-Banide}                          % Auteur
\date{20/04/2015}                              % Date de création
\begin{document}                               % Début du document
\maketitle{}                                   % Génération du titre

\section{Baroclinic Test case}   
 
~\\ This baroclinic instability test is being use on 3D atmospheric models instability test evolution. It is a pressure-based and height-based test. 

 \subsection{ Inisialization }
 ~\\ The tes case is inisialized with a constant surface pressure and with a surface geopotential equal to zero. The meridional wind is initialized as zero and will only be dependent of the velocity filed perturation. 

\subsection{Main equations}

~\\ As mention it is a presured and height based model.  The temperature equation is rules by:

\begin{equation*}
t=\frac{1}{ratio^2(\tau_1-\tau_2 \text{inttermT})} 
\end{equation*}

~\\ The pressure follow an hydrostatic rule define as :
\begin{equation}
p=p_0\exp(-\frac{g}{Rd}(\tau_{\text{int}_1}-\tau_{\text{int}_2}\text{intterm}_T))
\end{equation}
~\\ With $\text{intterm}_T$ representing the interior term on the temperature expression. 
\clearpage


~\\ The zonal velocity follow the following equation.  
\begin{equation}
 u=-\Omega_{ref} a_{ref} cos(\varphi)+\sqrt{(\Omega_{ref} a_{ref} cos(\varphi))^2+ a_{ref} \cos(\varphi)U)}
\end{equation}

A perturbation equation is applied to the velocity field. Depending of the model use stream fonction perturbation or exponential perturbation are applied on the the velocity field. A meridional velocity is noticed for stream function perturbation since it affect both zonal and meridional velocity field.













 
\subsection{Perturbations}


\begin{equation}
 q_{ratio}=q_0exp \left[-\frac{\phi}{\phi_{\omega}}^4 \right]exp \left[- \left(\frac{(\eta-1)p_0}{p_{\omega}} \right)^2\right]
\end{equation}

\begin{equation}
 q_{numb}=q_{ratio}q_{gc}\frac{P_{sv}}{p}\left[1-\frac{L_{vap}}{R_{vap}}\frac{\rho R_d}{p} + \frac{L_{vap}}{R_{vap} T_{ref}} \right]
\end{equation}

\begin{equation}
 q_{den}=\left[1+\frac{q_{ratio}q_{gc}P_{sv}}{p^2}\rho R_d M_{vap} \right]
\end{equation}

\begin{equation}
 q=q_{num}/q_{den}
\end{equation}

Great circle perturbation \\
\begin{equation*}
\text{Circle}_{\text{great}}=\frac{1}{\text{Pert}_{\text{expr}}}\text{arcos}(\sin(\text{Pert}_{\text{lat}}))\sin(\text{lat})+\cos(\text{Pert}_{\text{lat}})\cos(\text{lat})\cos(\text{lon}-\text{Pert}_{\text{lon}})
\end{equation*}




\clearpage 

\section{Tropical cyclone}

\subsection{ Inisialization }

~\\ 

\begin{equation}
p_s=p_0-\delta p \exp(-\frac{r_g}{r_p}^{n_r})
\end{equation}


\clearpage 
\appendix
\section{Annexe}

\begin{align*}
A&= \frac{1}{lapse} \\
B&=\frac{T_0-T_P}{T_0T_P} \\
C&=0.5(K+2) \frac{T_E-T_P}{T_ET_P} \\
H&=R_d\frac{T_0} {g}\\
Z&=\frac{z}{B H}  
\end{align*}


\begin{equation*}
T_0=0.5(T_E +T_P)
\end{equation*}

~\\Earth scalling parameter is used for the baroclinic instability, it initialization doesn't change from non scaled baroclinic instability. However the scalling parameter X as an effect on the the planet radius and it rotation rate. The impact on both of those parameter are the following $a_{ref}=a_{earth}/X$ and $\Omega_{ref}=\Omega{earth}X$ , nevertheless the product of the two is an changed 



\end{document}    